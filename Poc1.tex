\documentclass[a4paper, 12pt]{article}

\usepackage[T1]{fontenc}
\usepackage[utf8]{inputenc}
\usepackage[polish]{babel}
\usepackage{secdot}
\usepackage{geometry}
\usepackage{graphicx} 
\usepackage{float}
\usepackage{setspace}
\usepackage{hyperref}
\usepackage{standalone}
\usepackage{tikz}
\usepackage{tabularx}
\usepackage{epstopdf}
\usepackage{amsmath}
\usepackage{amssymb}
\usepackage{subfig}
\usepackage{enumerate}
\usepackage{titlesec}
\usepackage{underscore}
\usepackage{float}
\usepackage{mathtools}
\usepackage{setspace}
\usepackage{courier}
\usepackage{mathptmx}
\usepackage{seqsplit}
\usepackage[font={small,it}, labelfont={bf}]{caption}
\usepackage{listings}
\lstset{
  breaklines = true,
  literate={ą}{{\k a}}1
  		     {Ą}{{\k A}}1
           {ż}{{\. z}}1
           {Ż}{{\. Z}}1
           {ź}{{\' z}}1
           {Ź}{{\' Z}}1
           {ć}{{\' c}}1
           {Ć}{{\' C}}1
           {ę}{{\k e}}1
           {Ę}{{\k E}}1
           {ó}{{\' o}}1
           {Ó}{{\' O}}1
           {ń}{{\' n}}1
           {Ń}{{\' N}}1
           {ś}{{\' s}}1
           {Ś}{{\' S}}1
           {ł}{{\l}}1
           {Ł}{{\L}}1
}
%\usepackage[LGRgreek]{mathastext}

\graphicspath{{C:/Users/marci/Documents/GitHub/studia/POC/Lab_1/images/}}

\newgeometry{tmargin=2cm, bmargin=2cm, lmargin=2cm, rmargin=2cm}

\begin{document}
\begin{titlepage}
\newcommand{\ww}{0.32}
\begin{center}
\includegraphics[scale=1.1]{polsl_logo}
\bigskip

\Large{ \textbf{POLITECHNIKA ŚLĄSKA}

\textbf{WYDZIAŁ AUTOMATYKI, ELEKTRONIKI I INFORMATYKI} }
\bigskip
\bigskip
\bigskip
\bigskip

\textbf{Laboratorium}
\medskip

\textbf{Przetwarzanie Obrazów Cyfrowych}
\bigskip
\bigskip
\bigskip
\bigskip

\large{ \textbf{PREZENTACJA GRAFIKI RASTROWEJ - FORMATY PLIKÓW GRAFICZNYCH}}

\vspace{2cm}

\centering Ćwiczenie wykonano:\\
\centering08.04.2022

\vfill

\normalsize

\vspace{5cm}

\begin{flushleft}
\hspace{12.5cm}Autor: \\
\hspace{12.5cm}\textbf{Marcin Obyrtał} \\

\hspace{12.5cm}Semestr: VI \\

\hspace{12.5cm}Grupa: 2 \\

\end{flushleft}

\medskip

\vspace*{\fill}        Gliwice, 2022

\end{center}

\end{titlepage}

\section{Wybrane obrazy do testów}

\newcommand{\ww}{0.32}

\begin{figure}[h]

\captionsetup[subfloat]{position=bottom,labelformat=empty} 
\subfloat[Obraz rzeczywisty]{\includegraphics[width=\ww\linewidth]{pens.PNG}}
\hfill
\subfloat[Obraz rzeczywisty szary]{\includegraphics[width=\ww\linewidth]{woman_512x512.PNG}}
\hfill
\subfloat[Obraz syntetyczny]{\includegraphics[width=\ww\linewidth]{NCDvsIteraLenaGC.PNG}}\\

\caption{Wybrane obrazy do analizy}

\label{fig:fig1}
\end{figure}

\section{Wybrane wskaźniki}

\begin{enumerate}[a)]

  \item \textbf {PSNR} (peak signal-to-noise ratio) \--- stosunek maksymalnej mocy sygnału do mocy szumu zakłócającego ten sygnał. 
  Ze względu na szeroki zakres wartości PSNR wyrażany jest w decybelach. 
  W~ celu wyznaczenia PSNR należy wpierw obliczyć współczynnik MSE (ang. mean squared \linebreak error) bazujący na obu porównywanych obrazach, korzystając z wzoru:
  \begin{equation*}
  \begin{aligned}
    MSE =\frac{1}{N \cdot M}\sum_{i=1}^{N}\sum_{j=1}^{M}([f(i,j)-f'(i,j)]^2),
  \end{aligned}
  \end{equation*}
  gdzie:
  \begin{equation*}
  \begin{aligned}
    & N, M \textrm{\ --- wymiary obrazu w pikselach,} \\
    & f(i,j) \textrm{\ --- wartość piksela o współrzędnych } (i,j) \textrm{ obrazu oryginalnego,} \\
    & f'(i,j) \textrm{\ --- wartość piksela o współrzędnych } (i,j) \textrm{ obrazu skompresowanego.} \\
  \end{aligned}
  \end{equation*}
  Następnie wyliczoną wartość MSE należy podstawić do wzoru końcowego: \\
  \begin{equation*}
    \begin{aligned}
      & PSNR = 10 \cdot \mathrm{log_{10}}\frac{[\mathrm{max}(f(i,j))]^2}{MSE},
    \end{aligned}
  \end{equation*}
  gdzie:
  \begin{equation*}
  \begin{aligned}
    & \hspace{1.5cm}\textrm{max}(f(i,j)) \textrm{ --- wartość maksymalna danego sygnału; w przypadku obrazów zwykle}
  \end{aligned}
  \vspace{-0.3cm}
  \end{equation*}
  \hspace{1.55cm} jest to wartość stała, np. dla obrazów monochromatycznych o reprezentacji 8-bitowej \\
  \hspace*{1.5cm} wynosi 255.

  \item \textbf {MAE} (mean absolute error) \--- średnia z sumy błędów bezwględnych. Oblicza się ją korzystając ze wzoru:
  \begin{equation*}
    \begin{aligned}
      MAE =\frac{\sum_{i=1}^{n}|y_i - x_i|}{n},
    \end{aligned}
  \end{equation*}
    gdzie:
  \begin{equation*}
    \begin{aligned}
      & n \textrm{\ --- liczba pikseli w obrazie,} \\*
      & y_i, x_i \textrm{\ --- wartość piksela } i \textrm{-tego w obrazie oryginalnym i skompresowanym}
    \end{aligned}
  \end{equation*}
 
\end{enumerate}
  
\section{Podział formatów i opis GIFa}

\subsection{Podział rozszerzeń plików graficznych}

\begin{enumerate}

  \item grafika rastrowa
 \begin{enumerate}[a)]

  \item kompresja stratna
\begin{itemize}
  \item JPEG
  \item JPEG2000
  \item GIF
\end{itemize}
  \item kompresja bezstratna
\begin{itemize}
  \item PNG
  \item BMP
  \item TIFF
\end{itemize}
  \item brak kompresji
 \begin{itemize}
  \item XCF
\end{itemize}
\end{enumerate}

  \item grafika wektorowa
 \begin{itemize}
  \item SWF
  \item SVG 
\end{itemize}

\end{enumerate}

\subsection{Opis GIFa} 
\textbf{GIF} (ang. Graphics Interchange Format) to format pliku graficznego pozwalający również zapisywać animacje.
Format GIF jest formatem stratnym, gdyż pozwala na zapisie do 256 kolorów w danym bloku (często jest to równoznaczne
z całym plikiem). GIF jest oparty na paletach: kolory używane w~ obrazie (ramce) w pliku mają swoje wartości RGB zdefiniowane 
w tabeli palet, która może zawierać do 256 wpisów, a dane obrazu odnoszą się do kolorów za pomocą ich indeksów (0-255)   
w tabeli palet. 
Definicje kolorów w palecie mogą pochodzić z przestrzeni barw składającej się z milionów odcieni (224 odcienie, 8 bitów dla każdego 
koloru podstawowego), ale maksymalna liczba kolorów, z których może korzystać ramka, wynosi 256. 


\section{Przedstawienie metod kompresji obrazów}

W celu przedstawienia metod kompresji, wybrane obrazy zostały poddane zmianie rozszerzenia. Każdy obraz został zapisany do PNG, GIF,
JPG i JP2 (w przypadku dwóch ostatnich rozszerzeń został dobieranane 2 stopnie jakości (w przypadku JPG: 100 - minimalna kompresja,
0 - maksymalna kompresja), (w przypadku JP2: 1000 - minimalna kompresja, 0 - maksymalna  kompresja) w taki sposób, aby dla plików JPG
uzyskać obrazy: 
\begin{itemize}
\setlength\itemsep{-0.2cm}
 \item subiektywnie o wysokiej jakości przy jak najmniejszym rozmiarze
 \item subiektywnie pozwalający zobaczyć co przedstawia obrazek
\end{itemize}
Natomiast JP2 zostały dobrane tak, by były rozmiarowo zbliżone do JPG). \vspace*{-0.5cm}\\

\begin{flushleft}
Będziemy porównywać obrazy wybranymi wcześniej wskaźnikami jakości (PSNR i MAE). 
W~ przypadku PSNR, czym wynik większy, tym lepiej (lepsze odwzorowanie obrazu). 
W~ przypadku MAE, odwrotnie.
\end{flushleft}


\subsection{Obraz rzeczywisty}

\begin{figure}[H]
\captionsetup[subfloat]{position=bottom,labelformat=empty} 
\subfloat
[TIFF: \\
{Rozmiar: 617 [kB]} \\
PNG: \\
{Rozmiar: 237 [kB]} \\
{PSNR: INF [dB]} \\
{MAE: 0} \\
]
{\includegraphics[width=\ww\linewidth]{pens_frag.PNG}}
\hfill
\subfloat
[JPG: \\
$\mathrm{q_1 = 75}$ \\
{Rozmiar: 17,2 [kB]} \\
{PSNR: 32,0917092835493 [dB]} \\
{MAE: 4,06690045336537} \\
]
{\includegraphics[width=\ww\linewidth]{pens_75_frag.JPG}}
\hfill
\subfloat
[JP2: \\
$\mathrm{Q_1 = 55}$ \\
{Rozmiar: 16,8 [kB]} \\
{PSNR: 38,2927949147924 [dB]} \\
{MAE: 2,23602123454877} \\
]
{\includegraphics[width=\ww\linewidth]{pens_55_frag_jp2.PNG}}\\
\subfloat
[GIF: \\
{Rozmiar: 47,9 [kB]} \\
{PSNR: 34,0089220458643 [dB]}\\
{MAE: 2,83251314242906} \\
]
{\includegraphics[width=\ww\linewidth]{pens_frag_gif.PNG}}
\hfill
\subfloat
[JPG: \\
$\mathrm{q_2 = 8}$ \\
{Rozmiar: 4,84 [kB]} \\
{PSNR: 24,989539546534 [dB]}\\
MAE: 10,1340269435941 \\
]
{\includegraphics[width=\ww\linewidth]{pens_8_frag.JPG}}
\hfill
\subfloat
[JP2: \\
$\mathrm{Q_2 = 16}$ \\
{Rozmiar: 4,87 [kB]} \\
{PSNR: 28,9936454929455 [dB]}\\
MAE: 6,16604667984139 \\
]
{\includegraphics[width=\ww\linewidth]{pens_16_frag_jp2_v2.PNG}}

\caption{Porównanie metod kompresji dla obrazu rzeczywistego}

\label{fig:fig2}
\end{figure}

\textbf{Wnioski:} \\
W przypadku wybranego obrazu rzeczywistego:
\begin{itemize}
\setlength\itemsep{-0.2cm}
  \item Kompresja do PNG zmniejszyła rozmiar obrazka (ponad 2-krotnie), przy jednoczesnym 
  zachowaniu idealnego odwzorowania pikseli. 
  \item Kompresja do GIF znacząco zmniejszyła rozmiar pliku (ponad 12-krotnie),
   skala barw została lekko ograniczona
   \item Kompresja do JPG pozwoliła znacząco zmniejszyć rozmiar pliku (ponad 30-krotnie
   pozwalając uzyskać obraz, który mieści się w granicach tolerancji). Jeżeli zależy nam bardzo
   na jak najmniejszym rozmiarze pliku, ale tak, by informacja była możliwa do odczytania to jesteśmy
   w~ stanie uzyskać rozmiar aż 120-krotnie mniejszy niż oryginał.
   \item Kompresja do JP2 daje lepsze wyniki (lepsza jakość obrazu) w stosunku do JPG o tym samym rozmiarze.
\end{itemize}

\subsection{Obraz rzeczywisty szary}

\begin{figure}[H]
\captionsetup[subfloat]{position=bottom,labelformat=empty} 
\subfloat
[BMP: \\
{Rozmiar: 257 [kB]} \\
PNG: \\
{Rozmiar: 189 [kB]} \\
{PSNR: INF [dB]} \\
{MAE: 0} \\
]
{\includegraphics[width=\ww\linewidth]{woman_512x512_frag.PNG}}
\hfill
\subfloat
[JPG: \\
$\mathrm{q_1 = 85}$ \\
{Rozmiar: 56,3 [kB]} \\
{PSNR: 37,2757566658855 [dB]} \\
{MAE: 2,60620530565167} \\
]
{\includegraphics[width=\ww\linewidth]{woman_512x512_85_frag.JPG}}
\hfill
\subfloat
[JP2: \\
$\mathrm{Q_1 = 72}$ \\
{Rozmiar: 55,2 [kB]} \\
{PSNR: 39,8274037178062 [dB]} \\
{MAE: 2,02674740484429} \\
]
{\includegraphics[width=\ww\linewidth]{woman_512x512_72_frag_jp2.PNG}}\\
\subfloat
[GIF: \\
{Rozmiar: 222 [kB]} \\
{PSNR: INF [dB]} \\
{MAE: 0} \\
]
{\includegraphics[width=\ww\linewidth]{woman_512x512_frag_gif.PNG}}
\hfill
\subfloat
[JPG: \\
$\mathrm{q_2 = 13}$ \\
{Rozmiar: 11,2 [kB]} \\
{PSNR: 24,989539546534 [dB]}\\
MAE: 10,1340269435941 \\
]
{\includegraphics[width=\ww\linewidth]{woman_512x512_13_frag.JPG}}
\hfill
\subfloat
[JP2: \\
$\mathrm{Q_2 = 14}$ \\
{Rozmiar: 10,7 [kB]} \\
{PSNR: 31,5375477394586 [dB]}\\
MAE: 4,87587850826605 \\
]
{\includegraphics[width=\ww\linewidth]{woman_512x512_14_frag_jp2.PNG}}

\caption{Porównanie metod kompresji dla obrazu rzeczywistego szarego}

\label{fig:fig3}
\end{figure}

\textbf{Wnioski:} \\
W przypadku wybranego obrazu rzeczywistego szarego:

\begin{itemize}
  \setlength\itemsep{-0.2cm}
    \item Kompresja do PNG zmniejszyła rozmiar obrazka (o około 25\%), przy jednoczesnym 
    zachowaniu idealnego odwzorowania pikseli. 
    \item Kompresja do GIF praktycznie nie zmniejszyła rozmiaru obrazka (spadek o 12,5\%).
    Obraz nie uległ żadnym zmianom.
     \item Kompresja do JPG pozwoliła znacząco zmniejszyć rozmiar pliku (5-krotnie
     pozwalając uzyskać obraz, który mieści się w granicach tolerancji). Jeżeli zależy nam bardzo
     na jak najmniejszym rozmiarze pliku, ale tak, by informacja była możliwa do odczytania to jesteśmy 
     w~ stanie uzyskać rozmiar ponad 20-krotnie mniejszy niż oryginał.
     \item Kompresja do JP2 daje lepsze wyniki (lepsza jakość obrazu) w stosunku do JPG o tym samym rozmiarze.
  \end{itemize}

\subsection{Obraz syntetyczny}

\begin{figure}[H]
\captionsetup[subfloat]{position=bottom,labelformat=empty} 
\subfloat
[BMP: \\
{Rozmiar: 474 [kB]} \\
PNG: \\
{Rozmiar: 8,2 [kB]} \\
{PSNR: INF [dB]} \\
{MAE: 0} \\
]
{\includegraphics[width=\ww\linewidth]{NCDvsIteraLenaGC_frag.PNG}}
\hfill
\subfloat
[JPG: \\
$\mathrm{q_1 = 80}$ \\
{Rozmiar: 49,7 [kB]} \\
{PSNR: 29,4423531988225 [dB]} \\
{MAE: 1,41546793940652} \\
]
{\includegraphics[width=\ww\linewidth]{NCDvsIteraLenaGC_80_frag.JPG}}
\hfill
\subfloat
[JP2: \\
$\mathrm{Q_1 = 34}$ \\
{Rozmiar: 48,2 [kB]} \\
{PSNR: 44,42462315244512 [dB]} \\
{MAE: 0,550053261395864} \\
]
{\includegraphics[width=\ww\linewidth]{NCDvsIteraLenaGC_34_frag_jp2.PNG}}\\
\subfloat
[GIF: \\
{Rozmiar: 10,8 [kB]} \\
{PSNR: INF [dB]} \\
{MAE: 0} \\
]
{\includegraphics[width=\ww\linewidth]{NCDvsIteraLenaGC_frag_gif.PNG}}
\hfill
\subfloat
[JPG: \\
$\mathrm{q_2 = 10}$ \\
{Rozmiar: 18,4 [kB]} \\
{PSNR: 24,2226690767368 [dB]}\\
MAE: 3,4841516220516 \\
]
{\includegraphics[width=\ww\linewidth]{NCDvsIteraLenaGC_10_frag.JPG}}
\hfill
\subfloat
[JP2: \\
$\mathrm{Q_2 = 12}$ \\
{Rozmiar: 17 [kB]} \\
{PSNR: 29,528367743773 [dB]}\\
MAE: 2,62115307463512 \\
]
{\includegraphics[width=\ww\linewidth]{NCDvsIteraLenaGC_12_frag_jp2.PNG}}

\caption{Porównanie metod kompresji dla obrazu rzeczywistego szarego}

\label{fig:fig4}
\end{figure}

\begin{itemize}
  \setlength\itemsep{-0.2cm}
    \item Kompresja do PNG zmniejszyła rozmiar obrazka aż 60-krotnie, przy jednoczesnym 
    zachowaniu idealnego odwzorowania pikseli. 
    \item Kompresja do GIF praktycznie nie zmnieszyła rozmiaru obrazka ponad 40-krotnie.
    Obraz nie uległ żadnym zmianom.
     \item Kompresja do JPG pozwoliła znacząco zmniejszyć rozmiar pliku (9-krotnie
     pozwalając uzyskać obraz, który mieści się w granicach tolerancji). Jeżeli zależy nam bardzo
     na jak najmniejszym rozmiarze pliku, ale tak, by informacja była możliwa do odczytania to jesteśmy 
     w~ stanie uzyskać rozmiar 25-krotnie mniejszy niż oryginał.
     \item Kompresja do JP2 daje lepsze wyniki (lepsza jakość obrazu) w stosunku do JPG o tym samym rozmiarze.
  \end{itemize}

\section{Wykresy}

\subsection{Obraz rzeczywisty}

\begin{figure}[H]

\centering

{\includegraphics[width=15cm]{Wykres_re.PNG}}

\captionsetup{justification=centering}

\caption{Wykres zależności PSNR od wielkości pliku dla obrazu rzeczywistego \\ (rozszerzenie JPG i JP2)}

\label{fig:fig5}
\end{figure}

\textbf{Wnioski:} \\
W przypadku wybranego obrazu syntetycznego róznica w~ PSNR między rozszerzeniem JP2 a~ JPG nieznacznie rośnie wraz ze wzrostem 
rozmiaru pliku. 

\subsection{Obraz rzeczywisty szary}

\begin{figure}[H]

\centering
  
{\includegraphics[width=15cm]{Wykres_re_gray.PNG}}
  
\captionsetup{justification=centering}
  
\caption{Wykres zależności PSNR od wielkości pliku dla obrazu rzeczywistego szarego \\ (rozszerzenie JPG i JP2)}
  
\label{fig:fig6}
\end{figure}

\textbf{Wnioski:} \\ 
W przypadku obrazu rzeczywistego różnica w~ PSNR między rozszerzeniem JP2 i JPG maleje wraz ze wzrostem rozmiaru pliku.

\subsection{Obraz syntetyczny}

\begin{figure}[H]

\centering
  
{\includegraphics[width=15cm]{Wykres_syn.PNG}}
  
\captionsetup{justification=centering}
  
\caption{Wykres zależności PSNR od wielkości pliku dla obrazu syntetycznego \\ (rozszerzenie JPG i JP2)}
  
\label{fig:fig7}
\end{figure}

\textbf{Wnioski:} \\ 
W przypadku obrazu rzeczywistego różnica w~ PSNR między rozszerzeniem JP2 i JPG znacząco rośnie wraz ze wzrostem rozmiaru pliku.

\end{document}

